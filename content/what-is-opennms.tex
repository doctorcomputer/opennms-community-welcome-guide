\section{What is OpenNMS?}
\emph{OpenNMS} is a network management platform to give you the ability to solve problems in the FCAPS\footnote{ISO Telecommunications Management Network model and framework for Fault, Configuration, Accounting, Performance and Security categories} categories. The application gives you access to the management data through a web interface. The \emph{OpenNMS} project is a collaboration of developers and network management specialists around the world, to produce an open standard for a network management platform. The project aims to deliver a solution for all types of network management issues, massively scalable and feature rich. The technology consists of a series of interrelated programs delivering various components for a cloud infrastructure solution.

\begin{figure}[h]
	\centering
	\includegraphics[width=1.0\textwidth]{images/opennms-principles.png}
	\caption{OpenNMS parts and principles}
	\label{fig:opennms-principles}
\end{figure}

The platform of \emph{OpenNMS} today provides a platform for long term performance data collection, service assurance and integration points for your management infrastructure.

\subsection{OpenNMS Principles}
\begin{itemize}
  \item Open development model: All of the code of \emph{OpenNMS} is freely available under GNU Public License Version 3\footnote{GPLv3 license: \url{http://www.gnu.org/copyleft/gpl.html}}.
  \item Open development process: Every year the development community holds a developers conference to gather requirements and write specifications for the upcoming release.
  \item Open community: \emph{OpenNMS} is dedicated to producing a healthy, vibrant, and active developer and user community. Most decisions will be made using a lazy consensus model.
\end{itemize}

\subsection{OpenNMS Projects}
Network management has many different aspects of integration. Developing integration on configuration management databases or integrate vendor specific management architectures. Otherwise there are also technology specific topics, like different strategies to store and use time series data. All these different integrations projects are developed in the public git repository hosted on \url{https://www.github.com/opennms}.

\subsection{Release Process}
OpenNMS is currently on a \textbf{\textcolor{red}{???-month}} release cycle, which consists of \textbf{\textcolor{red}{x-many}} stages. Details on \textcolor{red}{\url{http://www.opennms.org/ReleaseCycle}}.

\subsubsection{Planning}
\textcolor{red}{Do we have a kind of a planning phase to decide which of the features come into a next release? Discussion, feedback to focus on the next release and lightweight spec writing.}

\subsubsection{Developers Conference}
\emph{The OpenNMS Group, Inc.} sponsors every year a developers conference - \emph{DevJam}. This conference brings all core developers, community members to work together on the project. The conference is also used to discuss concepts around the project. We track \emph{DevJam} topics in our wiki which can be found on \url{http://www.opennms.org/wiki/Dev-Jam}.

\subsubsection{JIRA}
\emph{OpenNMS} uses \emph{Atlassian JIRA} to track issues and for development planning. The development follows the agile \emph{Scrum} development process. Features are described as stories and will be developed in sprints. Details how we use \emph{JIRA} for the development process is documented in \textcolor{red}{\url{http://www.opennms.org/wiki/???}}

\subsubsection{Implementation}
The Implementation stage is split into a number of milestone iterations. The work in progress is published in a feature branch, which should then be proposed for merging when ready. \textcolor{red}{Code is proposed several weeks before each milestone release date so that it can be reviewed in a timely manner. Is it that way?!}

\subsubsection{Quality Assurance}
For quality assurance we use \emph{Atlassian Bamboo} as continuous integration system\footnote{Open Source Licensed, https://www.atlassian.com/software/bamboo}. It is publicly available on \url{http://bamboo.internal.opennms.com:8085}. \emph{Bamboo} compiles and runs tests for feature- and master branches. Additional we use \emph{SonarQube}\footnote{SonarQube website: \url{http://www.sonarqube.org/}} for statistic code analysis. \textcolor{red}{\url{Where-is-the-URL-to-Sonar-I-dont-know?!}}. There are also branches intended to fix bugs and do not introduce new features. This branches are named after the \emph{JIRA} issue followed the pattern NMS-\{issue-number\}. Builds are automatically triggered with committing to a feature- or master branch.

\subsubsection{Release}
Before an stable version is released, we have a release candidate freeze (RCF). This happens a \textbf{\textcolor{red}{x-days/weeks/months}} before an actual \textbf{\textcolor{red}{Release Day - do we have one?!}}. \emph{OpenNMS} releases are numbered using the \textbf{\textcolor{red}{YYYY.N time based scheme (our detailed scheme here, I have seen we changed with Bamboo our release scheme a bit.)}}. For example, \textbf{\textcolor{red}{an example follows here}}. The release is identified using a codename. Code names are \textbf{\textcolor{red}{Is there a pattern for our code names?}} \textbf{\textcolor{red}{Code names are chosen by people in the \emph{IRC} channel or whatever :) More details on \url{https://www.opennms.org/wiki/Release_Naming???}}}.

\subsection{Governance}
The \emph{OpenNMS} project is governed by \textbf{\textcolor{red}{your group here - OGPs + OpenNMS Group?!}}. \textbf{\textcolor{red}{Some ideas how OpenStack organized it, Foundation board of directors, technical committee, user comittee and a wiki page \url{http://wiki.openstack.org/Governance/Foundation/Bylaws}}}
